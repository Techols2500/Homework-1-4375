% Options for packages loaded elsewhere
\PassOptionsToPackage{unicode}{hyperref}
\PassOptionsToPackage{hyphens}{url}
%
\documentclass[
]{article}
\usepackage{amsmath,amssymb}
\usepackage{lmodern}
\usepackage{iftex}
\ifPDFTeX
  \usepackage[T1]{fontenc}
  \usepackage[utf8]{inputenc}
  \usepackage{textcomp} % provide euro and other symbols
\else % if luatex or xetex
  \usepackage{unicode-math}
  \defaultfontfeatures{Scale=MatchLowercase}
  \defaultfontfeatures[\rmfamily]{Ligatures=TeX,Scale=1}
\fi
% Use upquote if available, for straight quotes in verbatim environments
\IfFileExists{upquote.sty}{\usepackage{upquote}}{}
\IfFileExists{microtype.sty}{% use microtype if available
  \usepackage[]{microtype}
  \UseMicrotypeSet[protrusion]{basicmath} % disable protrusion for tt fonts
}{}
\makeatletter
\@ifundefined{KOMAClassName}{% if non-KOMA class
  \IfFileExists{parskip.sty}{%
    \usepackage{parskip}
  }{% else
    \setlength{\parindent}{0pt}
    \setlength{\parskip}{6pt plus 2pt minus 1pt}}
}{% if KOMA class
  \KOMAoptions{parskip=half}}
\makeatother
\usepackage{xcolor}
\IfFileExists{xurl.sty}{\usepackage{xurl}}{} % add URL line breaks if available
\IfFileExists{bookmark.sty}{\usepackage{bookmark}}{\usepackage{hyperref}}
\hypersetup{
  pdftitle={Homework 1},
  pdfauthor={Tyler Echols},
  hidelinks,
  pdfcreator={LaTeX via pandoc}}
\urlstyle{same} % disable monospaced font for URLs
\usepackage[margin=1in]{geometry}
\usepackage{color}
\usepackage{fancyvrb}
\newcommand{\VerbBar}{|}
\newcommand{\VERB}{\Verb[commandchars=\\\{\}]}
\DefineVerbatimEnvironment{Highlighting}{Verbatim}{commandchars=\\\{\}}
% Add ',fontsize=\small' for more characters per line
\usepackage{framed}
\definecolor{shadecolor}{RGB}{248,248,248}
\newenvironment{Shaded}{\begin{snugshade}}{\end{snugshade}}
\newcommand{\AlertTok}[1]{\textcolor[rgb]{0.94,0.16,0.16}{#1}}
\newcommand{\AnnotationTok}[1]{\textcolor[rgb]{0.56,0.35,0.01}{\textbf{\textit{#1}}}}
\newcommand{\AttributeTok}[1]{\textcolor[rgb]{0.77,0.63,0.00}{#1}}
\newcommand{\BaseNTok}[1]{\textcolor[rgb]{0.00,0.00,0.81}{#1}}
\newcommand{\BuiltInTok}[1]{#1}
\newcommand{\CharTok}[1]{\textcolor[rgb]{0.31,0.60,0.02}{#1}}
\newcommand{\CommentTok}[1]{\textcolor[rgb]{0.56,0.35,0.01}{\textit{#1}}}
\newcommand{\CommentVarTok}[1]{\textcolor[rgb]{0.56,0.35,0.01}{\textbf{\textit{#1}}}}
\newcommand{\ConstantTok}[1]{\textcolor[rgb]{0.00,0.00,0.00}{#1}}
\newcommand{\ControlFlowTok}[1]{\textcolor[rgb]{0.13,0.29,0.53}{\textbf{#1}}}
\newcommand{\DataTypeTok}[1]{\textcolor[rgb]{0.13,0.29,0.53}{#1}}
\newcommand{\DecValTok}[1]{\textcolor[rgb]{0.00,0.00,0.81}{#1}}
\newcommand{\DocumentationTok}[1]{\textcolor[rgb]{0.56,0.35,0.01}{\textbf{\textit{#1}}}}
\newcommand{\ErrorTok}[1]{\textcolor[rgb]{0.64,0.00,0.00}{\textbf{#1}}}
\newcommand{\ExtensionTok}[1]{#1}
\newcommand{\FloatTok}[1]{\textcolor[rgb]{0.00,0.00,0.81}{#1}}
\newcommand{\FunctionTok}[1]{\textcolor[rgb]{0.00,0.00,0.00}{#1}}
\newcommand{\ImportTok}[1]{#1}
\newcommand{\InformationTok}[1]{\textcolor[rgb]{0.56,0.35,0.01}{\textbf{\textit{#1}}}}
\newcommand{\KeywordTok}[1]{\textcolor[rgb]{0.13,0.29,0.53}{\textbf{#1}}}
\newcommand{\NormalTok}[1]{#1}
\newcommand{\OperatorTok}[1]{\textcolor[rgb]{0.81,0.36,0.00}{\textbf{#1}}}
\newcommand{\OtherTok}[1]{\textcolor[rgb]{0.56,0.35,0.01}{#1}}
\newcommand{\PreprocessorTok}[1]{\textcolor[rgb]{0.56,0.35,0.01}{\textit{#1}}}
\newcommand{\RegionMarkerTok}[1]{#1}
\newcommand{\SpecialCharTok}[1]{\textcolor[rgb]{0.00,0.00,0.00}{#1}}
\newcommand{\SpecialStringTok}[1]{\textcolor[rgb]{0.31,0.60,0.02}{#1}}
\newcommand{\StringTok}[1]{\textcolor[rgb]{0.31,0.60,0.02}{#1}}
\newcommand{\VariableTok}[1]{\textcolor[rgb]{0.00,0.00,0.00}{#1}}
\newcommand{\VerbatimStringTok}[1]{\textcolor[rgb]{0.31,0.60,0.02}{#1}}
\newcommand{\WarningTok}[1]{\textcolor[rgb]{0.56,0.35,0.01}{\textbf{\textit{#1}}}}
\usepackage{graphicx}
\makeatletter
\def\maxwidth{\ifdim\Gin@nat@width>\linewidth\linewidth\else\Gin@nat@width\fi}
\def\maxheight{\ifdim\Gin@nat@height>\textheight\textheight\else\Gin@nat@height\fi}
\makeatother
% Scale images if necessary, so that they will not overflow the page
% margins by default, and it is still possible to overwrite the defaults
% using explicit options in \includegraphics[width, height, ...]{}
\setkeys{Gin}{width=\maxwidth,height=\maxheight,keepaspectratio}
% Set default figure placement to htbp
\makeatletter
\def\fps@figure{htbp}
\makeatother
\setlength{\emergencystretch}{3em} % prevent overfull lines
\providecommand{\tightlist}{%
  \setlength{\itemsep}{0pt}\setlength{\parskip}{0pt}}
\setcounter{secnumdepth}{-\maxdimen} % remove section numbering
\ifLuaTeX
  \usepackage{selnolig}  % disable illegal ligatures
\fi

\title{Homework 1}
\usepackage{etoolbox}
\makeatletter
\providecommand{\subtitle}[1]{% add subtitle to \maketitle
  \apptocmd{\@title}{\par {\large #1 \par}}{}{}
}
\makeatother
\subtitle{4375 Machine Learning with Dr.~Mazidi}
\author{Tyler Echols}
\date{6/2/2022}

\begin{document}
\maketitle

This homework has two parts:

\begin{itemize}
\tightlist
\item
  Part 1 uses R for data exploration
\item
  Part 2 uses C++ for data exploration
\end{itemize}

\begin{center}\rule{0.5\linewidth}{0.5pt}\end{center}

This homework is worth 100 points, 50 points each for Part 1 and Part 2.

\begin{center}\rule{0.5\linewidth}{0.5pt}\end{center}

\hypertarget{part-1-rstudio-data-exploration}{%
\section{Part 1: RStudio Data
Exploration}\label{part-1-rstudio-data-exploration}}

\textbf{Instructions:} Follow the instructions for the 10 parts below.
If the step asks you to make an observation or comment, write your
answer in the white space above the gray code box for that step.

\hypertarget{step-1-load-and-explore-the-data}{%
\subsection{Step 1: Load and explore the
data}\label{step-1-load-and-explore-the-data}}

\begin{itemize}
\tightlist
\item
  load library MASS (install at console, not in code)
\item
  load the Boston dataframe using data(Boston)
\item
  use str() on the data
\item
  type ?Boston at the console
\item
  Write 2-3 sentences about the data set below
\end{itemize}

Your commentary here: \# For the housing values of boston it is divided
into a list with 506 row, and 14 columns. The data also contains various
abriviated terms for eacch assingned value given to the terms of the
houses representing the price value of the reported number

\begin{Shaded}
\begin{Highlighting}[]
\CommentTok{\# step 1 code}
\FunctionTok{library}\NormalTok{(MASS)}
\FunctionTok{data}\NormalTok{(}\StringTok{"Boston"}\NormalTok{)}
\FunctionTok{str}\NormalTok{(Boston)}
\end{Highlighting}
\end{Shaded}

\begin{verbatim}
## 'data.frame':    506 obs. of  14 variables:
##  $ crim   : num  0.00632 0.02731 0.02729 0.03237 0.06905 ...
##  $ zn     : num  18 0 0 0 0 0 12.5 12.5 12.5 12.5 ...
##  $ indus  : num  2.31 7.07 7.07 2.18 2.18 2.18 7.87 7.87 7.87 7.87 ...
##  $ chas   : int  0 0 0 0 0 0 0 0 0 0 ...
##  $ nox    : num  0.538 0.469 0.469 0.458 0.458 0.458 0.524 0.524 0.524 0.524 ...
##  $ rm     : num  6.58 6.42 7.18 7 7.15 ...
##  $ age    : num  65.2 78.9 61.1 45.8 54.2 58.7 66.6 96.1 100 85.9 ...
##  $ dis    : num  4.09 4.97 4.97 6.06 6.06 ...
##  $ rad    : int  1 2 2 3 3 3 5 5 5 5 ...
##  $ tax    : num  296 242 242 222 222 222 311 311 311 311 ...
##  $ ptratio: num  15.3 17.8 17.8 18.7 18.7 18.7 15.2 15.2 15.2 15.2 ...
##  $ black  : num  397 397 393 395 397 ...
##  $ lstat  : num  4.98 9.14 4.03 2.94 5.33 ...
##  $ medv   : num  24 21.6 34.7 33.4 36.2 28.7 22.9 27.1 16.5 18.9 ...
\end{verbatim}

\hypertarget{step-2-more-data-exploration}{%
\subsection{Step 2: More data
exploration}\label{step-2-more-data-exploration}}

Use R commands to:

\begin{itemize}
\tightlist
\item
  display the first few rows
\item
  display the last two rows
\item
  display row 5
\item
  display the first few rows of column 1 by combining head() and using
  indexing
\item
  display the column names
\end{itemize}

\begin{Shaded}
\begin{Highlighting}[]
\CommentTok{\# step 2 code}
\FunctionTok{head}\NormalTok{(Boston)}
\end{Highlighting}
\end{Shaded}

\begin{verbatim}
##      crim zn indus chas   nox    rm  age    dis rad tax ptratio  black lstat
## 1 0.00632 18  2.31    0 0.538 6.575 65.2 4.0900   1 296    15.3 396.90  4.98
## 2 0.02731  0  7.07    0 0.469 6.421 78.9 4.9671   2 242    17.8 396.90  9.14
## 3 0.02729  0  7.07    0 0.469 7.185 61.1 4.9671   2 242    17.8 392.83  4.03
## 4 0.03237  0  2.18    0 0.458 6.998 45.8 6.0622   3 222    18.7 394.63  2.94
## 5 0.06905  0  2.18    0 0.458 7.147 54.2 6.0622   3 222    18.7 396.90  5.33
## 6 0.02985  0  2.18    0 0.458 6.430 58.7 6.0622   3 222    18.7 394.12  5.21
##   medv
## 1 24.0
## 2 21.6
## 3 34.7
## 4 33.4
## 5 36.2
## 6 28.7
\end{verbatim}

\begin{Shaded}
\begin{Highlighting}[]
\FunctionTok{tail}\NormalTok{(Boston, }\AttributeTok{n=}\DecValTok{2}\NormalTok{)}
\end{Highlighting}
\end{Shaded}

\begin{verbatim}
##        crim zn indus chas   nox    rm  age    dis rad tax ptratio  black lstat
## 505 0.10959  0 11.93    0 0.573 6.794 89.3 2.3889   1 273      21 393.45  6.48
## 506 0.04741  0 11.93    0 0.573 6.030 80.8 2.5050   1 273      21 396.90  7.88
##     medv
## 505 22.0
## 506 11.9
\end{verbatim}

\begin{Shaded}
\begin{Highlighting}[]
\NormalTok{Boston[}\DecValTok{5}\NormalTok{,]}
\end{Highlighting}
\end{Shaded}

\begin{verbatim}
##      crim zn indus chas   nox    rm  age    dis rad tax ptratio black lstat
## 5 0.06905  0  2.18    0 0.458 7.147 54.2 6.0622   3 222    18.7 396.9  5.33
##   medv
## 5 36.2
\end{verbatim}

\begin{Shaded}
\begin{Highlighting}[]
\FunctionTok{head}\NormalTok{(Boston[,}\DecValTok{1}\NormalTok{])}
\end{Highlighting}
\end{Shaded}

\begin{verbatim}
## [1] 0.00632 0.02731 0.02729 0.03237 0.06905 0.02985
\end{verbatim}

\begin{Shaded}
\begin{Highlighting}[]
\FunctionTok{names}\NormalTok{(Boston)}
\end{Highlighting}
\end{Shaded}

\begin{verbatim}
##  [1] "crim"    "zn"      "indus"   "chas"    "nox"     "rm"      "age"    
##  [8] "dis"     "rad"     "tax"     "ptratio" "black"   "lstat"   "medv"
\end{verbatim}

\hypertarget{step-3-more-data-exploration}{%
\subsection{Step 3: More data
exploration}\label{step-3-more-data-exploration}}

For the crime column, show:

\begin{itemize}
\tightlist
\item
  the mean
\item
  the median
\item
  the range
\end{itemize}

\begin{Shaded}
\begin{Highlighting}[]
\CommentTok{\# step 3 code}
\FunctionTok{mean}\NormalTok{(crimtab)}
\end{Highlighting}
\end{Shaded}

\begin{verbatim}
## [1] 3.246753
\end{verbatim}

\begin{Shaded}
\begin{Highlighting}[]
\FunctionTok{median}\NormalTok{(crimtab)}
\end{Highlighting}
\end{Shaded}

\begin{verbatim}
## [1] 0
\end{verbatim}

\begin{Shaded}
\begin{Highlighting}[]
\FunctionTok{range}\NormalTok{(crimtab)}
\end{Highlighting}
\end{Shaded}

\begin{verbatim}
## [1]  0 58
\end{verbatim}

\hypertarget{step-4-data-visualization}{%
\subsection{Step 4: Data
visualization}\label{step-4-data-visualization}}

Create a histogram of the crime column, with an appropriate main
heading. In the space below, state your conclusions about the crime
variable:

Your commentary here: \# With the hist feature it can create a histogram
of any data it's reading in. If you want it to be more specific in how
detailed you want the numbers to be displayed you can tell it to
specifically read from 1 column.

\begin{Shaded}
\begin{Highlighting}[]
\CommentTok{\# step 4 code}
\FunctionTok{hist}\NormalTok{(crimtab, }\AttributeTok{main=}\StringTok{"Histogram for Boston Crime Per Capita by Town"}\NormalTok{ )}
\end{Highlighting}
\end{Shaded}

\includegraphics{Homework_1_files/figure-latex/unnamed-chunk-4-1.pdf}

\hypertarget{step-5-finding-correlations}{%
\subsection{Step 5: Finding
correlations}\label{step-5-finding-correlations}}

Use the cor() function to see if there is a correlation between crime
and median home value. In the space below, write a sentence or two on
what this value might mean. Also write about whether or not the crime
column might be useful to predict median home value.

Your commentary here: \# The correlation between crim and medv is -0.30.
meaning that the rate is a a negative linear relationship

\begin{Shaded}
\begin{Highlighting}[]
\CommentTok{\# step 5 code}
\FunctionTok{cor}\NormalTok{(Boston}\SpecialCharTok{$}\NormalTok{crim, Boston}\SpecialCharTok{$}\NormalTok{medv)}
\end{Highlighting}
\end{Shaded}

\begin{verbatim}
## [1] -0.3883046
\end{verbatim}

\hypertarget{step-6-finding-potential-correlations}{%
\subsection{Step 6: Finding potential
correlations}\label{step-6-finding-potential-correlations}}

Create a plot showing the median value on the y axis and number of rooms
on the x axis. Create appropriate main, x and y labels, change the point
color and style. {[}Reference for
plots(\url{http://www.statmethods.net/advgraphs/parameters.html})

Use the cor() function to quantify the correlation between these two
variables. Write a sentence or two summarizing what the graph and
correlation tell you about these 2 variables.

Your commentary here: \# It shows that a majority are in the 20 range
where there are houses with 6 of less rooms.

\begin{Shaded}
\begin{Highlighting}[]
\CommentTok{\# step 6 code}
\FunctionTok{plot}\NormalTok{(medv }\SpecialCharTok{\textasciitilde{}}\NormalTok{ rm, }\AttributeTok{data =}\NormalTok{ Boston, }\AttributeTok{main =} \StringTok{" Price median and \# of rooms "}\NormalTok{, }\AttributeTok{xlab =} \StringTok{"Number of rooms"}\NormalTok{, }\AttributeTok{ylab =} \StringTok{"Median house price"}\NormalTok{ , }\AttributeTok{col =} \FunctionTok{rainbow}\NormalTok{ (}\DecValTok{69}\NormalTok{) , }\AttributeTok{pch =} \DecValTok{17}\NormalTok{ , }\AttributeTok{col.lab =} \StringTok{"Blue"}\NormalTok{)}
\end{Highlighting}
\end{Shaded}

\includegraphics{Homework_1_files/figure-latex/unnamed-chunk-6-1.pdf}

\begin{Shaded}
\begin{Highlighting}[]
\FunctionTok{cor}\NormalTok{(Boston}\SpecialCharTok{$}\NormalTok{medv, Boston}\SpecialCharTok{$}\NormalTok{rm)}
\end{Highlighting}
\end{Shaded}

\begin{verbatim}
## [1] 0.6953599
\end{verbatim}

\hypertarget{step-7-evaluating-potential-predictors}{%
\subsection{Step 7: Evaluating potential
predictors}\label{step-7-evaluating-potential-predictors}}

Use R functions to determine if variable chas is a factor. Plot median
value on the y axis and chas on the x axis. Make chas a factor and plot
again.

Comment on the difference in meaning of the two graphs. Look back the
description of the Boston data set you got with the ?Boston command to
interpret the meaning of 0 and 1.

Your commentary here: \# The fact it seems that both are at perfect
points on the graph with no in the middle valuse. They are all either
0.0 or 0.1.

\begin{Shaded}
\begin{Highlighting}[]
\CommentTok{\# step 7 code}

\FunctionTok{plot}\NormalTok{(medv }\SpecialCharTok{\textasciitilde{}}\NormalTok{ chas, }\AttributeTok{data =}\NormalTok{ Boston, }\AttributeTok{main =} \StringTok{" Price median and Charles River dummy variable "}\NormalTok{, }\AttributeTok{xlab =} \StringTok{"Charles River dummy variable"}\NormalTok{, }\AttributeTok{ylab =} \StringTok{"Median house price"}\NormalTok{ , }\AttributeTok{col =} \FunctionTok{rainbow}\NormalTok{ (}\DecValTok{69}\NormalTok{) , }\AttributeTok{pch =} \DecValTok{17}\NormalTok{ , }\AttributeTok{col.lab =} \StringTok{"Blue"}\NormalTok{)}
\end{Highlighting}
\end{Shaded}

\includegraphics{Homework_1_files/figure-latex/unnamed-chunk-7-1.pdf}

\begin{Shaded}
\begin{Highlighting}[]
\FunctionTok{is.factor}\NormalTok{(Boston}\SpecialCharTok{$}\NormalTok{chas)}
\end{Highlighting}
\end{Shaded}

\begin{verbatim}
## [1] FALSE
\end{verbatim}

\begin{Shaded}
\begin{Highlighting}[]
\NormalTok{?Boston }
\end{Highlighting}
\end{Shaded}

\begin{verbatim}
## starting httpd help server ... done
\end{verbatim}

\hypertarget{step-8-evaluating-potential-predictors}{%
\subsection{Step 8: Evaluating potential
predictors}\label{step-8-evaluating-potential-predictors}}

Explore the rad variable. What kind of variable is rad? What information
do you get about this variable with the summary() function? Does the
unique() function give you additional information? Use the sum()
function to determine how many neighborhoods have rad equal to 24. Use R
code to determine what percentage this is of the neighborhoods.

Your commentary here: \# You get information about how accessible it is
to each radial highway. Also listing the various numbers of neighbors

\begin{Shaded}
\begin{Highlighting}[]
\CommentTok{\# step 8 code }
\FunctionTok{summary}\NormalTok{(Boston}\SpecialCharTok{$}\NormalTok{rad)}
\end{Highlighting}
\end{Shaded}

\begin{verbatim}
##    Min. 1st Qu.  Median    Mean 3rd Qu.    Max. 
##   1.000   4.000   5.000   9.549  24.000  24.000
\end{verbatim}

\begin{Shaded}
\begin{Highlighting}[]
\FunctionTok{unique}\NormalTok{(Boston}\SpecialCharTok{$}\NormalTok{rad)}
\end{Highlighting}
\end{Shaded}

\begin{verbatim}
## [1]  1  2  3  5  4  8  6  7 24
\end{verbatim}

\begin{Shaded}
\begin{Highlighting}[]
\FunctionTok{sum}\NormalTok{(Boston}\SpecialCharTok{$}\NormalTok{rad)}
\end{Highlighting}
\end{Shaded}

\begin{verbatim}
## [1] 4832
\end{verbatim}

\hypertarget{step-9-adding-a-new-potential-predictor}{%
\subsection{Step 9: Adding a new potential
predictor}\label{step-9-adding-a-new-potential-predictor}}

Create a new variable called ``far'' using the ifelse() function that is
TRUE if rad is 24 and FALSE otherwise. Make the variable a factor. Plot
far and medv. What does the graph tell you?

Your commentary here: \# This graph tells me that everything is high on
when you look at 0 for far. But when you look at the one for 1.0 it
seeems like it is more twoard the lower 30's and 20's.

\begin{Shaded}
\begin{Highlighting}[]
\CommentTok{\# step 9 code }
\FunctionTok{ifelse}\NormalTok{(Boston}\SpecialCharTok{$}\NormalTok{rad }\SpecialCharTok{==} \DecValTok{24}\NormalTok{, far}\OtherTok{\textless{}{-}} \ConstantTok{TRUE}\NormalTok{, far}\OtherTok{\textless{}{-}}\ConstantTok{FALSE}\NormalTok{)}
\end{Highlighting}
\end{Shaded}

\begin{verbatim}
##   [1] FALSE FALSE FALSE FALSE FALSE FALSE FALSE FALSE FALSE FALSE FALSE FALSE
##  [13] FALSE FALSE FALSE FALSE FALSE FALSE FALSE FALSE FALSE FALSE FALSE FALSE
##  [25] FALSE FALSE FALSE FALSE FALSE FALSE FALSE FALSE FALSE FALSE FALSE FALSE
##  [37] FALSE FALSE FALSE FALSE FALSE FALSE FALSE FALSE FALSE FALSE FALSE FALSE
##  [49] FALSE FALSE FALSE FALSE FALSE FALSE FALSE FALSE FALSE FALSE FALSE FALSE
##  [61] FALSE FALSE FALSE FALSE FALSE FALSE FALSE FALSE FALSE FALSE FALSE FALSE
##  [73] FALSE FALSE FALSE FALSE FALSE FALSE FALSE FALSE FALSE FALSE FALSE FALSE
##  [85] FALSE FALSE FALSE FALSE FALSE FALSE FALSE FALSE FALSE FALSE FALSE FALSE
##  [97] FALSE FALSE FALSE FALSE FALSE FALSE FALSE FALSE FALSE FALSE FALSE FALSE
## [109] FALSE FALSE FALSE FALSE FALSE FALSE FALSE FALSE FALSE FALSE FALSE FALSE
## [121] FALSE FALSE FALSE FALSE FALSE FALSE FALSE FALSE FALSE FALSE FALSE FALSE
## [133] FALSE FALSE FALSE FALSE FALSE FALSE FALSE FALSE FALSE FALSE FALSE FALSE
## [145] FALSE FALSE FALSE FALSE FALSE FALSE FALSE FALSE FALSE FALSE FALSE FALSE
## [157] FALSE FALSE FALSE FALSE FALSE FALSE FALSE FALSE FALSE FALSE FALSE FALSE
## [169] FALSE FALSE FALSE FALSE FALSE FALSE FALSE FALSE FALSE FALSE FALSE FALSE
## [181] FALSE FALSE FALSE FALSE FALSE FALSE FALSE FALSE FALSE FALSE FALSE FALSE
## [193] FALSE FALSE FALSE FALSE FALSE FALSE FALSE FALSE FALSE FALSE FALSE FALSE
## [205] FALSE FALSE FALSE FALSE FALSE FALSE FALSE FALSE FALSE FALSE FALSE FALSE
## [217] FALSE FALSE FALSE FALSE FALSE FALSE FALSE FALSE FALSE FALSE FALSE FALSE
## [229] FALSE FALSE FALSE FALSE FALSE FALSE FALSE FALSE FALSE FALSE FALSE FALSE
## [241] FALSE FALSE FALSE FALSE FALSE FALSE FALSE FALSE FALSE FALSE FALSE FALSE
## [253] FALSE FALSE FALSE FALSE FALSE FALSE FALSE FALSE FALSE FALSE FALSE FALSE
## [265] FALSE FALSE FALSE FALSE FALSE FALSE FALSE FALSE FALSE FALSE FALSE FALSE
## [277] FALSE FALSE FALSE FALSE FALSE FALSE FALSE FALSE FALSE FALSE FALSE FALSE
## [289] FALSE FALSE FALSE FALSE FALSE FALSE FALSE FALSE FALSE FALSE FALSE FALSE
## [301] FALSE FALSE FALSE FALSE FALSE FALSE FALSE FALSE FALSE FALSE FALSE FALSE
## [313] FALSE FALSE FALSE FALSE FALSE FALSE FALSE FALSE FALSE FALSE FALSE FALSE
## [325] FALSE FALSE FALSE FALSE FALSE FALSE FALSE FALSE FALSE FALSE FALSE FALSE
## [337] FALSE FALSE FALSE FALSE FALSE FALSE FALSE FALSE FALSE FALSE FALSE FALSE
## [349] FALSE FALSE FALSE FALSE FALSE FALSE FALSE FALSE  TRUE  TRUE  TRUE  TRUE
## [361]  TRUE  TRUE  TRUE  TRUE  TRUE  TRUE  TRUE  TRUE  TRUE  TRUE  TRUE  TRUE
## [373]  TRUE  TRUE  TRUE  TRUE  TRUE  TRUE  TRUE  TRUE  TRUE  TRUE  TRUE  TRUE
## [385]  TRUE  TRUE  TRUE  TRUE  TRUE  TRUE  TRUE  TRUE  TRUE  TRUE  TRUE  TRUE
## [397]  TRUE  TRUE  TRUE  TRUE  TRUE  TRUE  TRUE  TRUE  TRUE  TRUE  TRUE  TRUE
## [409]  TRUE  TRUE  TRUE  TRUE  TRUE  TRUE  TRUE  TRUE  TRUE  TRUE  TRUE  TRUE
## [421]  TRUE  TRUE  TRUE  TRUE  TRUE  TRUE  TRUE  TRUE  TRUE  TRUE  TRUE  TRUE
## [433]  TRUE  TRUE  TRUE  TRUE  TRUE  TRUE  TRUE  TRUE  TRUE  TRUE  TRUE  TRUE
## [445]  TRUE  TRUE  TRUE  TRUE  TRUE  TRUE  TRUE  TRUE  TRUE  TRUE  TRUE  TRUE
## [457]  TRUE  TRUE  TRUE  TRUE  TRUE  TRUE  TRUE  TRUE  TRUE  TRUE  TRUE  TRUE
## [469]  TRUE  TRUE  TRUE  TRUE  TRUE  TRUE  TRUE  TRUE  TRUE  TRUE  TRUE  TRUE
## [481]  TRUE  TRUE  TRUE  TRUE  TRUE  TRUE  TRUE  TRUE FALSE FALSE FALSE FALSE
## [493] FALSE FALSE FALSE FALSE FALSE FALSE FALSE FALSE FALSE FALSE FALSE FALSE
## [505] FALSE FALSE
\end{verbatim}

\begin{Shaded}
\begin{Highlighting}[]
\NormalTok{far }\OtherTok{\textless{}{-}} \FunctionTok{ifelse}\NormalTok{(Boston}\SpecialCharTok{$}\NormalTok{rad }\SpecialCharTok{==} \DecValTok{24}\NormalTok{, far}\OtherTok{\textless{}{-}} \ConstantTok{TRUE}\NormalTok{, far}\OtherTok{\textless{}{-}}\ConstantTok{FALSE}\NormalTok{)}
\FunctionTok{plot}\NormalTok{(medv }\SpecialCharTok{\textasciitilde{}}\NormalTok{ far, }\AttributeTok{data =}\NormalTok{ Boston,  }\AttributeTok{xlab =} \StringTok{"far"}\NormalTok{, }\AttributeTok{ylab =} \StringTok{"Medv"}\NormalTok{ , }\AttributeTok{col =} \FunctionTok{rainbow}\NormalTok{ (}\DecValTok{69}\NormalTok{) , }\AttributeTok{pch =} \DecValTok{17}\NormalTok{ , }\AttributeTok{col.lab =} \StringTok{"Blue"}\NormalTok{)}
\end{Highlighting}
\end{Shaded}

\includegraphics{Homework_1_files/figure-latex/unnamed-chunk-9-1.pdf}

\hypertarget{step-10-data-exploration}{%
\subsection{Step 10: Data exploration}\label{step-10-data-exploration}}

\begin{itemize}
\tightlist
\item
  Create a summary of Boston just for columns 1, 6, 13 and 14 (crim, rm,
  lstat, medv)
\item
  Use the which.max() function to find the neighborhood with the highest
  median value. See p.~176 in the pdf
\item
  Display that row from the data set, but only columns 1, 6, 13 and 14
\item
  Write a few sentences comparing this neighborhood and the city as a
  whole in terms of: crime, number of rooms, lower economic percent,
  median value.
\end{itemize}

Your commentary here: \# it looks as if the crime rate is low starting
out but arounf the 3rd quater it picked up. the number of rooms per
house had a slight increase but has seemed to hit a high value with the
max number of rooms in houses to be high. with the increase of of
population maxing out at 37.97. So that about have the ammount of people
living in this city which must mean this city is to expensive to live
in.

\begin{Shaded}
\begin{Highlighting}[]
\CommentTok{\# step 10 code}
\FunctionTok{summary}\NormalTok{(Boston}\SpecialCharTok{$}\NormalTok{crim) }
\end{Highlighting}
\end{Shaded}

\begin{verbatim}
##     Min.  1st Qu.   Median     Mean  3rd Qu.     Max. 
##  0.00632  0.08204  0.25651  3.61352  3.67708 88.97620
\end{verbatim}

\begin{Shaded}
\begin{Highlighting}[]
\FunctionTok{summary}\NormalTok{(Boston}\SpecialCharTok{$}\NormalTok{rm)}
\end{Highlighting}
\end{Shaded}

\begin{verbatim}
##    Min. 1st Qu.  Median    Mean 3rd Qu.    Max. 
##   3.561   5.886   6.208   6.285   6.623   8.780
\end{verbatim}

\begin{Shaded}
\begin{Highlighting}[]
\FunctionTok{summary}\NormalTok{(Boston}\SpecialCharTok{$}\NormalTok{lstat)}
\end{Highlighting}
\end{Shaded}

\begin{verbatim}
##    Min. 1st Qu.  Median    Mean 3rd Qu.    Max. 
##    1.73    6.95   11.36   12.65   16.95   37.97
\end{verbatim}

\begin{Shaded}
\begin{Highlighting}[]
\FunctionTok{summary}\NormalTok{(Boston}\SpecialCharTok{$}\NormalTok{medv)}
\end{Highlighting}
\end{Shaded}

\begin{verbatim}
##    Min. 1st Qu.  Median    Mean 3rd Qu.    Max. 
##    5.00   17.02   21.20   22.53   25.00   50.00
\end{verbatim}

\begin{Shaded}
\begin{Highlighting}[]
\NormalTok{x }\OtherTok{\textless{}{-}} \FunctionTok{which.max}\NormalTok{(Boston}\SpecialCharTok{$}\NormalTok{medv)}
\FunctionTok{print}\NormalTok{(Boston[x,}\FunctionTok{c}\NormalTok{(}\StringTok{"crim"}\NormalTok{, }\StringTok{"rm"}\NormalTok{, }\StringTok{"lstat"}\NormalTok{,}\StringTok{"medv"}\NormalTok{ )])}
\end{Highlighting}
\end{Shaded}

\begin{verbatim}
##        crim    rm lstat medv
## 162 1.46336 7.489  1.73   50
\end{verbatim}

\hypertarget{part-2-c}{%
\section{Part 2: C++}\label{part-2-c}}

In this course we will get some experience writing machine learning
algorithms from scratch in C++, and comparing performance to R. Part 2
of Homework 1 is designed to lay the foundation for writing custom
machine learning algorithms in C++.

To complete Part 2, first you will read in the Boston.csv file which
just contains columns rm and medv.

\begin{center}\rule{0.5\linewidth}{0.5pt}\end{center}

In the C++ IDE of your choice:

1 Read the csv file (now reduced to 2 columns) into 2 vectors of the
appropriate type. See the reading in cpp picture in Piazza.

2 Write the following functions:

\begin{itemize}
\item
  a function to find the sum of a numeric vector
\item
  a function to find the mean of a numeric vector
\item
  a function to find the median of a numeric vector
\item
  a function to find the range of a numeric vector
\item
  a function to compute covariance between rm and medv (see formula on
  p.~74 of pdf)
\item
  a function to compute correlation between rm and medv (see formula on
  p.~74 of pdf); Hint: sigma of a vector can be calculated as the square
  root of variance(v, v)
\end{itemize}

3 Call the functions described in a-d for rm and for medv. Call the
covariance and correlation functions. Print results for each function.

\end{document}
